  \chapter{Third-Party Code and Libraries}
%TC:ignore


%If you have made use of any third party code or software libraries, i.e. any code that you have not designed and written yourself, then you must include this appendix.

%As has been said in lectures, it is acceptable and likely that you will make use of third-party code and software libraries. The key requirement is that we understand what is your original work and what work is based on that of other people.

%Therefore, you need to clearly state what you have used and where the original material can be found. Also, if you have made any changes to the original versions, you must explain what you have changed.

%As an example, you might include a definition such as:

%Apache POI library � The project has been used to read and write Microsoft Excel files (XLS) as part of the interaction with the client�s existing system for processing data. Version 3.10-FINAL was used. The library is open source and it is available from the Apache Software Foundation
%\cite{apache_poi}. The library is released using the Apache License
%\cite{apache_license}. This library was used without modification.

The following section will discuss the dependencies used on or during the project. It will not label all the the libraries dependencies, as they were not modified.

\textbf{Flask} - The library has been used as the main framework of the web application. The library was used for routing, and all interactions with the application. Version \textit{0.10.1} was used \cite{citeulike:13160396}. Released under three clause BSD License \cite{citeulike:14025861}. This library was used without modification.

\textbf{Flask-SQLAlchemy} - The library was used as an ORM tool to interact with the database layer via insertion, deletion of data. Version \textit{2.1} was used \cite{citeulike:14025864}. Released under three clause BSD License \cite{citeulike:14025861}. This library was used without modification.

\textbf{python-dateutil} - The project integrates with dates and the library provides a clear way to parse dates in an easy manner \cite{citeulike:14025869}. Version \textit{2.5.3} was used. Released under three clause BSD License \cite{citeulike:14025861}. This library was used without modification.

\textbf{google-api-python-client} - The project required interactions with the Google Services. Google suggest using this library for Python application to attempt to query their APIs and return associated data. The Google Plus and Google Calendar APIs were used \cite{citeulike:14025877}. Version \textit{1.5.0} was used. Released under the Apache License, Version 2.0 \cite{apache_license}.  This library was used without modification.

\textbf{OpenCV} - The project integrated with OpenCV libraries to segment the image and extract the text from an image \cite{citeulike:13206865}. The library was used for the binarisation and smoothing of noise on the image. Used when the user uploads their image. Using Version \textit{3.0.0}. Released under three clause BSD License \cite{citeulike:14025861}. This library was used without modification.

\textbf{Tesseract} - The project required handwritten notes to be analysed by an OCR tool . The Tesseract tool was used for training the user's handwriting as well as used to identify the characters on the page \cite{citeulike:14014368}. It also integrates with the \textit{tesserocr} library to parse the three lines of text. Version \textit{3.04.00} was used. Released under the Apache License, Version 2.0 \cite{apache_license}. The library was used without modification.

\textbf{TesserOCR} - The application needed to integrate with the Tesseract engine. Initially the Tesseract command line interface was going to be used \cite{citeulike:14021437}. However, the confidence scores could not be identified. Version \textit{2.0.0} was used. Released under the MIT license \cite{citeulike:14025880}. The library was used without modification.

\textbf{Flask-SeaSurf} - The application needed to be prevented against CSRF and Flask did not support it automatically so a library was used to add security to the system \cite{citeulike:14025881}. Version \textit{2.1} was used. Released under three clause BSD License \cite{citeulike:14025861}. The library was used without modification.

\textbf{Numpy} - Used throughout the OpenCV implementation for interactions with the image arrays. Used as a tool to ease the modification of the images \cite{citeulike:14025883}. Version \textit{1.11.0} was used. Released under three clause BSD License \cite{citeulike:14025861}. The library was used without modification.

\textbf{Matplotlib} - The library was used to create a series of graphs which would help to show the segmented image, especially compared to the original image \cite{Hunter:2007}. Version \textit{1.5.1} was used. Released under Matplotlib license \cite{citeulike:14025887}. The library was used without modification.

\textbf{JQuery} - jQuery is a common library in web development. Used to ease the interactions with the DOM, it was useful for accessing fields on the webpage \cite{citeulike:14025897}. Version \textit{1.10.2} was used. Released under the MIT licence \cite{citeulike:14025880}. The library was used without modification.

\textbf{JQuery Datepicker} - The library offers default styling to jQuery build elements. Used to implement the datepicker into the add and edit metadata form \cite{citeulike:14025900}. Version \textit{1.11.4} was used. Released under the MIT license \cite{citeulike:14025880}. The library was used without modification.

\textbf{jQuery TimePicker} - The library is build upon jQuery to provide the functionality to add a datepicker to the forms so the user could select the times for their lecture \cite{citeulike:14004013}. Version \textit{1.8.11} was used. Released under the MIT license \cite{citeulike:14025880}. The library was used without modification.

\textbf{PIL} - The Python image library was used for the Exif data parsing, where the filename would be converted into an image \cite{citeulike:14024992}. Version \textit{3.2.0} was used. Released under the MIT license \cite{citeulike:14025880}. The library was used without modification.

\textbf{PyTest} - The testing framework is used to run the automated tests and provide an interface to create tests \cite{citeulike:1402058}. Version \textit{2.8.7} was used. Released under the MIT license \cite{citeulike:14025880}. The library was used without modification.

\textbf{Selenium} - Acceptance tests were an important part of the application development. Selenium tests ensured that items could be selected from the webpage with ease \cite{citeulike:14020625}. Version \textit{2.52.0} was used. Released under the creative commons 4.0 International license \cite{citeulike:14025914}. The library was used without modification.

\textbf{Flask-testing} - Offered better testing support for Flask applications. Made use of extending the classes from LiveServerTest and TestCase to ease with the testing infrastructure \cite{citeulike:14020588}. Version \textit{0.4.2} was used. Released under three clause BSD License \cite{citeulike:14025861}. The library was used without modification.

\textbf{oauth2client} - Since OAuth2 was needed to be implemented, Google suggested the library for reliability issues. Used to connect the services via OAuth \cite{citeulike:14025877}. Version \textit{2.0.1} was used. Released under the Apache License, Version 2.0 \cite{apache_license}. The library was used without modification.

\textbf{httplib2} - HTTP requests would need to be made to the external services, as a result the library provided an interface for using HTTP requests to the service \cite{citeulike:14025936}. Version \textit{0.9.2} was used.  Released under the MIT license \cite{citeulike:14025880}. The library was used without modification.

\textbf{jTessBoxEditor} - Editing the training data examples proved to be a cumbersome task so an additional graphical editor was used to help identify the characters and their boxes \cite{citeulike:13926798}. Version \textit{v1.4} was used.  Released under the Apache License, Version 2.0 \cite{apache_license}. The library was used without modification.

\textbf{Mock} - The mock library was an imperative library used throughout the application. During testing a lot of the data needed to be mocked and this library provided both patch functions \cite{citeulike:14020599}. Version \textit{1.3.0} was used. Released under three clause BSD License \cite{citeulike:14025861}. The library was used without modification.

\textbf{Flask-Migrate} - To ensure that the database could be migrated properly then Flask-Migrate was used so specific versions of the versions could be created \cite{citeulike:14025941}. Version \textit{1.8.0} was used.  Released under the MIT license \cite{citeulike:14025880}. The library was used without modification.

\textbf{Flask-Script} - Used as a part of the migration script so specific scripts could be enabled \cite{citeulike:14025943}. Version \textit{2.0.5} was used. Released under three clause BSD License \cite{citeulike:14025861}. The library was used without modification.

\textbf{Werkzeug} - Although this is installed as a dependency through Flask, the application makes use of the secure filename function in the uploading routing \cite{citeulike:14025945}. Version \textit{0.11.3} was used. Released under three clause BSD License \cite{citeulike:14025861}. The library was used without modification.

\textbf{ImageMagick} - Although not used in the end for the image processing it was used as prototyping work early on in the project. It was not used in the end \cite{citeulike:14023816}. Version \textit{6.9.2-4} was used. Released under ImageMagick license \cite{citeulike:14027133}. The Library was used without modification.

\textbf{Pip} - The package manager which helped to organise and install dependencies on the project \cite{citeulike:14025946}. Version \textit{8.1.1}. Released under the MIT licence \cite{citeulike:14025880}. The library was used without modification.

\textbf{PostgreSQL} - Used as the RDBMS of choice throughout the project. Version \textit{9.5.1} was used. Released under PostgreSQL license \cite{citeulike:14027320}. Library was used without modification.

\textbf{SQLite} - Used for the testing database. Version \textit{3.8.10.2}. License can be found under the public domain \cite{citeulike:14027312}. Library was used without modification.

%TC:endignore
