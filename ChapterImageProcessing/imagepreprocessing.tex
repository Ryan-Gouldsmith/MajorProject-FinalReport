\chapter{Image pre-processing for Tesseract}
%Image magick section
\section{ImageMagick}
In order to produce the best possible results with Tesseract, there had to be a pre-processing stage beforehand. Initially, the image was converted to grayscale using ImageMagick[CITE] but this showed that training was difficult and was not yielding the success rates which was needed.

Pursuing using ImageMagick for the pre-processing stage the image was converted to Monochrome. This managed to be able to pick up more characters in the image, but in doing so picked up too much noise around the image. This distorted the characters and made it hard to recognise characters with the correct confidence that Tesseract required.

Dr Hannah Dee suggested that maybe OpenCV[cite] image manipulation would be useful for the pre-processing step, instead of using ImageMagick.

% Open CV
\section{OpenCV}
% give a background of why we are using open cv
OpenCV is an open source image processing library. There was a choice of using the C++ or the Python bindings. Due to the applications core infrastructure it was agreed that Python would be easier to implement.

An investigation began by looking into the OTSU binarisation method and other versions of the adaptive thresholds. The need for a binarisation method, instead of using the default image, is that it reduced the ambiguity of the light sources on the image and any shadows which may occur during the capturing of the image.

After doing some investigations into different thresholding algorithms it came to light that Tesseract OCR uses OTSU thresholding as its underlying image processing algorithm. This quickly explained why a lot of characters were not being identified from images, when looking at the output of the experiment that were conducted. From this it could be seen that using OTSU as a pre-processing step for Tesseract would not return the best results.

Looking at the experiment of different thresholding techniques it was clear that adaptive mean and adaptive gaussian returned the image clearly binarised. There is no shadow or dark patch over the image and the text on the page is legible.

\section{Line removal}
After using Tesseract to train my handwriting on plain paper it showed that some sentences it could parse accurately, but those which are on a slant or slightly skewed, because of the lack of lines, returned a worse recognition rates.

To try and straighten the lines of text normal lined paper was used, but quickly found this was not good when binarising it as it often left unconnected lines which were represented as a series of dots. If the erode function was used on the image, it would lose the quality of the characters correctly identified.

This led to a creation of my own lined paper, one which would have thick lines and one which the lines could be filterable. The premise of using the blue lined paper was to correctly identify the blue lines and remove them from the image, leaving uniform lines of text which could be analysed. This process went through a couple of iterations.

To begin with it was decided to just extract the text from the image which fell between a black to gray threshold colour - excluding the blue lines from the image. This was followed up by the morphological operation, erode, to remove some of the remaining pixels left over by the lines. This worked to an extent and showed the binarised image with some loss of pixels on the characters and some were completely undistinguishable.

Although this worked well, it wasn't the most suitable solution as often line pixels were left in the image and would be picked up by the OCR resulting in an erroneous output from the system. So, again, it went through another iteration.

The next solution would binarise the image more elegantly and would actually remove the lines from the image and clear up the image to just leave the image binarised, with no blue or black lines.

[CITE REFERENCE USED TO DO THIS]
Firstly, it would read the image as grayscale and then apply a median blur to the image. This was needed to try and ease the noise and smooth the image. Much like the first iteration adaptive threshold with a gaussian mean was used to binarise the image. Afterwards, a mask was collected containing the horizontal black lines, using the structuring elements ``MORPH\_RECT''. This was eroded and dilated to try and remove the lines as much as possible. Intermediate masks so that black text was passed onto a new canvas - aiming to get as much of the text from the images across as a possible. This naturally carried some of the disjointed lines across, and suffered the same issues the previous iteration had to deal with. However, this iteration includes the use of finding contours.

The contours were considered using because a way was needed to find connected components and filter out the lines. Naturally, if the lines have been dilated and eroded then they will not be connected closely. Therefore, by choosing connected components on the image it will correctly get the characters from the image. Once the connected components of the characters were identified these were transferred to a blank mask, with a final erosion to clear up the image.

This technique has worked well in removing the blue lines from the page, whilst keeping the character quality. It clearly shows a binarised image with the text on the page clearly identified, and it worked on multiple examples. What is more impressing is that because of the adaptive threshold and different morphological operations then it can clearly identify characters in terribly lit photos. 
