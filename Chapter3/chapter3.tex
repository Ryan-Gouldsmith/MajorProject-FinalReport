\chapter{Implementation}

%The implementation should look at any issues you encountered as you tried to implement your design. During the work, you might have found that elements of your design were unnecessary or overly complex; perhaps third party libraries were available that simplified some of the functions that you intended to implement. If things were easier in some areas, then how did you adapt your project to take account of your findings?

%It is more likely that things were more complex than you first thought. In particular, were there any problems or difficulties that you found during implementation that you had to address? Did such problems simply delay you or were they more significant?

%You can conclude this section by reviewing the end of the implementation stage against the planned requirements.

\section{Image processing}

\section{Handwriting Training}
During the start of the handwriting training phase problems occurred such as it not reading the characters from a greyscale image correctly.
\subsection{Training process}
When using training data to be worked with Tesseract, it requires it to be in a specific file format. Such as: $<lang>.<font>.exp<number>.tiff$. When ran through a language, Tesseract outputs a box file which contains on each line: the character and the coordinates of this box. Trying to analyse this box was almost impossible. On the Tesseract wiki page [CITE] there's a link to jTessBoxEditor [CITE]. This tool was used to tag the boxes with their associated content.

Whilst using this editor the boxes could be expanded or shrunk to give the best possible fit to the characters. This was often utilised due to erroneous characters picked up by the editor.

% add some more stuff about training here.




\section{Web application}

\subsection{OAuth}

\subsection{Reoccuring events}
Reocurring events were discovered as an issue in the pre-beta user testing. During the design phase when thinking about the calendar, it was forgotten that reocurring events and all-day events could exist.

All-day events do not have the dateTime key response from the Google Calendar API. As a result the code would fail when trying to access the dateTime from the event start date. This resulted in a redesign and re-think of the possible issues which could arise from Google Calendar.

Eventually, the dateTime key was checked and the all day events issue was solved.

However, the reocurring events problem was still existing. When querying for an event, if the event was reocurring then it would group the reocurring events by the first time in which the event was created. This resulted in an image, which was taken on the 12th March 2016 for example, to show events in February - if there was a reocurring event on the 12th March. However, it had an important reoccurence event ID key.

This resulted in a further query being created which would return all the instances that were reocurring. This had to pass in the $event['id']$ to the query to return all these instances; it was filtered down by querying for the start and end date.

When editing a reocurring event, Google calendar performs some unexpected behaviour: instead of silently modifying the event and returning the grouped event, again, it instead returns both the grouped event and the edited event. A succinct solution for this has not been found and has been a slight issue.

\subsection{Tesseract Confidence}
During a meeting with Dr Hannah Dee, it was suggested that some form of confidence score couldbe outputted to the user to show how well Tesseract identifies the text from the image.

The Tesseract command line does not output the confidence of the characters identified; only the C++ library can output the confidence. Due to time constraints, a wrapper for the C++ Tesseract API could not be implemented - so a third party library was chosen, tessocr[CITE].

Tessocr offered the implementation to access the confidence values for the associated words. The algorithm to execute the identification of the characters was quite simple.

Firstly we get all the text lines; Tesseract deals with the lines as a series of text lines. This is then enumerated over and for each line a corresponding list of confidence words is collected from tehe $map_all_words$ API.

Due to this returning a tuple which is an immutable object in Python, modifications on the list could not be made easily. This resulted in the view file checking the tuple content and calculating whether it is above the threshold; 75 for green, 70 for orange and below 65 for red.

\subsection{Displaying calendar events}

\subsection{Parsing Exif data}

\subsection{Editing calendar events}



\section{Review against the requirements}
