\chapter{Evaluation}

%Examiners expect to find in your dissertation a section addressing such questions as:

%\begin{itemize}
%   \item Were the requirements correctly identified?
%   \item Were the design decisions correct?
%   \item Could a more suitable set of tools have been chosen?
%   \item How well did the software meet the needs of those who were expecting to use it?
%   \item How well were any other project aims achieved?
%   \item If you were starting again, what would you do differently?
%\end{itemize}

%Such material is regarded as an important part of the dissertation; it should demonstrate that you are capable not only of carrying out a piece of work but also of thinking critically about how you did it and how you might have done it better. This is seen as an important part of an honours degree.

%There will be good things and room for improvement with any project. As you write this section, identify and discuss the parts of the work that went well and also consider ways in which the work could be improved.

%Review the discussion on the Evaluation section from the lectures. A recording is available on Blackboard.

\section{Correctly identified requirements}
To recap, the following requirements and objectives were identified at the beginning of the project:
\begin{enumerate}
	\item Investigate how to extract handwriting text from an image - this will involve looking into ways OCR tools can interpret handwriting.
	\item Train the OCR to recognise text of the author's handwriting.
	\item Produce a set of a rules which a note must comply to.
	\item Produce a web application to form the core part of the product. This includes allowing a user to upload an image, display the image. Add appropriate tagging to a note such as module code.
	\item The user must be able to search for a given module code, showing the fill list of notes based on the module code entered.
	\item The backend of the application must conduct basic OCR recognition, analysing the first 3 lines of the notes.
	\item The backend must integrate with a calendar to archive the notes away later to be found again.
\end{enumerate}

The afformentioned requirements were part of the core system requirements; these were classufied as the minimum which the application must do. Personally, these targets, partly self-imposed, were extremeley challenging. The amount of work that was taken on, with lots of different components required that work was produced at a steady pace.

Investigation and research work was extensively conducted to identify how handwriting can be extract from an image. This was further enhanced by looking to optimise the performance of the Tesseract engine. Not only was the image binarised - there were studies and iterations of improvemet which would help to benefit the end user.

After twelve different training examples, from hand, the Tesseract enging is not yielding any greater of a success. Therefore it can be concluded that a sufficient job has been achieved with handwriting training. Probably one of the most frustrating experiences with labelling individual characters. Nevertheless, the training could not return a greater success than currently therefore good work has been completed to get it to that stage.

With the web application being the core part of the problem, then a lot of effort was centered in on ensuring the correct functionality was added. Personally, the application reflects that of the work carried out on it. A user can upload an image to the application and tag all associated meta-data, including the title, module code, date, and location.

The application creates an easy to use interface that allows the user to search for a given module code with ease. The case in which they enter the text is not important, as the application handles the case-sensitivity allowing for a better user experience.

One of the more impressing features of the application is the handwriting recognition integration. It parses the first three lines of the image and shows it to the user in an intuititve manner. This was built ontop of the work already implemented in the project with the handwriting recognition. With the final steps included the application has its true purpose, and that extra bit of complexity which personally the application needed.

Finally, from the original requirements the calendar integration was succesfully implemented; it was in retrospect one of the more challenging tasks to overcome. Ranging from the testing the services to implementing reocurring events, Google calendar always threw up a lot of issues. The reocurring events and all day events were particially frustrating, however there has been a wide interaction with the calendar. It can list all events in the last 7 days, find specific events by day and is able to find reocurring events. Ontop of this is can add to an event description and remove a specific string from the description field.

Overall, the application produced meets all the core requirements and goes beyond, in some aspects, the original features. The app has been well tested with a strong testing infrastructure backing the application. There have been complex design decisions to ensure that the design of the application is simple and intuitive to a reader. Personally, the app has been completed to a good standard.

\section{Design decisions}

\section{Use of tools}
\subsection{Flask}
During the project there were times in which there were feelings that the toolset should be different. Flask, the micro-framework for the web application, was pehaps a bad decision in hindsight. There was a lack of documentation and out-the-box support made even the most basic of functionality difficult and uncessarily complex. The configuration files for example: there was no default support for multiple support for default configuration files.

Another example is default security concerns. As stressed the importance of in \textit{Internet based applications} cross-site request forjery (CSRF) is a big issue among internet security. By default Flask does not support CSRF checking. An additional 3rd party library, SeaSurf, had to be installed to handle the basic functionality of this.

That said, Flask did offer a great deal of flexibility regarding the project. Directories were customisable easily, and the use of blueprints enabled that MVC feel to the project. There were times which just the basic support for a framework would have aided the developer and helped to improve the productivity, rather than being subjected to more external code.
\subsection{OpenCV}
The choice of using OpenCV compared to ImageMagick proved to be an extremely important design change. This helped the Tesseract training to increase substaintially, without the use of OpenCV it cpuld be argued that the Tesseract training would never have gotten off the ground. The research conducted into looking into different binarisation techniques was imperitive; there could have been a long chase going down the greyscale image route, trying to train the data and not getting anywhere.

\subsection{PostgreSQL}
The use of PostgreSQL was overall a good choice. Potentially all the decisions regarding using that over MySQL were not fully met, restrospecitively analysising the decision.  However, the decision to use that over SQLlite, especially as more than one person would use it would be a wise decision that was right to be made.

However, one aspect which should be noted was that if the project was to start again then a more serious look into the NoSQL database should be considered. During the application development, it was acknowledged by Dr Hannah Dee that it would be good to use the application for conferences too. The conferences do not follow the same structure as that of a lecture, prodominetly lecture name and module code.

Due to the variability in the differing data returned for different event then prehaps a NoSQL approach would have been beneficial.

\subsection{Tesseract}
Tesseract itself has been a great tool during the creation of the product. The training process was a little tedious but the tool would not be swapped out for another form of OCR tool.

\subsection{Google Calendar}
Although Google Calendar threw up a lot of issues along the way, the choice was a good decision in the end. After analysing the calendars choosing the most popular one appealed to more of the market.

Additionally, all the good support that Google gave [CITE Issue] was beneficial in the end, due to a large community of developers available.
\section{Meeting the users needs}
\section{Additional project aims}
Due to the nature of the project, all the features and functionality could not be implented into one project spanning just 15 weeks. Due to this a few requirements were cut back from the initial specification.

There are tasks such as: auto-skewing and cropping the image on upload, search by date, share link with other peers, extracting images from the text and training it on generic handwriting.  These features would have made the application a lot more useable for the user.
ADD SOME MORE
\section{Starting again}
If this project was to start again, there would be a stronger emphasis on the design considerations for the bigger picture. As previously stated, a mongo database would have been chosen over a RDMS, offering more flexibilty and scope for the application to become a viable success.
