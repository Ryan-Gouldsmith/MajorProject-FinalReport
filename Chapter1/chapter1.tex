\chapter{Background \& Objectives}

%This section should discuss your preparation for the project, including background reading, your analysis of the problem and the process or method you have followed to help structure your work.  It is likely that you will reuse part of your outline project specification, but at this point in the project you should have more to talk about.

%\textbf{Note}:

%\begin{itemize}
   %\item All of the sections and text in this example are for illustration purposes. The main Chapters are a good starting point, but the content and actual sections that you include are likely to be different.

  % \item Look at the document on the Structure of the Final Report for additional guidance.

%\end {itemize}

\section{Background}
%What was your background preparation for the project? What similar systems did you assess? What was your motivation and interest in this project?
Handwriting notes is still considered to be an important aspect of note taking. In [SMOKER CITE] a study was conducted and out of 61 adults 72.1\% prefered to take notes using pen and paper, rather than on a computer. Smoker et al [Cite]  concluded that recall rates for handwritten text was greater than that of typed text - prooving handwriting notes are viable.

However technology has advanced and we're moving into an era where we track and view everything digitally, from email to calendar entries. Therefore, there is a need to transfer the productivity from handwritten notes to digital notes - so they can be located easily.

\subsection{Similar Systems}
There are a number of systems which offer similar, and sometimes the same functionality that MapMyNotesApplication would use. The systems are
\begin{itemize}
  \item OneNote
  \item EverNote
\end{itemize}
Each of these tools offer something slightly different and it would be good to produce a system that would encapse these different aspects.
\subsubsection{OneNote}
OneNote is a note-taking and organisational app made by Microsoft [cite].
\subsubsection{EverNote}
EverNote is a note-taking and note organisation app, it is both supported on the web and in app form. EverNote is widely used and would provide a lot of the functionality a user would need to upload their note. They have realeased development articles [CITE-2013 one] stating that they are able to do OCR recognition on images. This would allow the user to upload an image and it would give a list of words, which it thinks is the word identified. This differs from MapMyNotes where the author aims to develop an application which would give a 1-1 word comparision, rather than a list of words.


\subsection{Why this project?}
The author's motivation for this project is to provide a tool, which initially would be tailored to lecturer based note content. Often notes are written up, but discarded into a big pile and later they're hard to find. As a user, it would be good if the notes were all in one place that they could photograph in and it would automatically create meta-data for it and save it to their calendar. Often the modules that are undertaken are stored in my calendar as a reminder that there's a lecturer on that day - so providing a link to the event in the calendar would be a nice way to tie all these dispersed information together. 

\section{Analysis}
%Taking into account the problem and what you learned from the background work, what was your analysis of the problem? How did your analysis help to decompose the problem into the main tasks that you would undertake? Were there alternative approaches? Why did you choose one approach compared to the alternatives?

%There should be a clear statement of the objectives of the work, which you will evaluate at the end of the work.

%In most cases, the agreed objectives or requirements will be the result of a compromise between what would ideally have been produced and what was felt to be possible in the time available. A discussion of the process of arriving at the final list is usually appropriate.
\begin{itemize}
  \item Chatted with Hannah with what was available in the time frame.
  \item I looked into different OCR tools to see how I could analyse different characters
  \item Looked into different ways to binarise an image accordingly.
  \item Created a list of user stories after the discussion with hannah to decompose the problems.
  \item I chose to do OCR recognition for singular pages and hopefully getting th etop 3 lines as meta-data than normal notes because I needed structure.
  \item the objective was to produce a web application which would allow a user to upload the image to the website nd it would suggest automatic tagging and integrate with a google calendar.
  \item I wanted to do the whole thing as OCR but a compremise would be that I include OCR and image segemntation into the dissertation as initially it was just going to be a web application.
\end{itemize}

\section{Process}
%You need to describe briefly the life cycle model or research method that you used. You do not need to write about all of the different process models that you are aware of. Focus on the process model that you have used. It is possible that you needed to adapt an existing process model to suit your project; clearly identify what you used and how you adapted it for your needs.
The project has been completed using an Agile approach. More specifically, a scrum [TODO CITE] approach has been used along with Extreme Programming [cite]. In the scrum methodology sprints are used to identify tasks and user stories to complete. At the end of every sprint a review and retrospective was conducted to analyse the performance of a sprint; this highlighted any issues with the process for the given sprint as well as thinking about improving this to be more effective next sprint.

\begin{itemize}
  \item Story points per sprint analysis.
  \item Taiga.io
  \item Planning and restrospectives with weekly blog posts
\end{itemize}

As well as Scrum, many Extreme Programming principles was used.
\begin{itemize}
  \item Continuous Integration
  \item TDD
  \item Refactoring
  \item YAGNI
  \item CRC cards

\end{itemize}
