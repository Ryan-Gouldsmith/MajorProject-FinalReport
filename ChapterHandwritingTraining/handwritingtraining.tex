\chapter{Handwriting training with Tesseract}

[CITE THAT GUYS THESIS]

Handwriting recognition is still an active research project and one which is constantly evolving. There are many complexities with handwriting such as whether to analyse cursive or non-cursive text. The project could have been taken in a couple of direction: write my own handwriting recognition system or use an OCR tool. Since the premise of the application is to provide a software tool for a user creating my own handwriting recognition system is not viable. As a result, an OCR tool has been used.

Consideration went into the different OCR tools out there, with commercial vs non-commercial and there was one open source technology which seemed very reliable, Tesseract.[CITE PAPER ON COMPARISON].

Tesseract is an open source C++ library for analysing handwriting, the current stable version at the time of writing the report is 3.04. It is mainly interacted through a command line application, and that has what has been used in this project.

\section{Training}

To begin to analyse the characters on a page, there was a decision that the training process would involve analysing non-cursive handwriting. Although this limits the user experience, analysing handwriting is a challenging task in itself, so by making it non-cursive there was a better chance of good recognition.

Secondly, another caveat with training is that the training process would be consisted on the author's handwriting, and not on the general public. This again, limits the application to the user, but that could be expanded in the future.

In order to successfully, train the handwriting then the pre-processing steps described in chapter 2 are required. Outputted from the image pre-processing is a tiff file - which then used for the Tesseract training.

Firstly, the image has to be in the following format: <language>.<font>.exp<expnumber>.tiff, for example the following file would be valid: eng.ryan.exp1a.tiff.
